
% Default to the notebook output style

    


% Inherit from the specified cell style.




    
\documentclass[11pt]{article}

    
    
    \usepackage[T1]{fontenc}
    % Nicer default font (+ math font) than Computer Modern for most use cases
    \usepackage{mathpazo}

    % Basic figure setup, for now with no caption control since it's done
    % automatically by Pandoc (which extracts ![](path) syntax from Markdown).
    \usepackage{graphicx}
    % We will generate all images so they have a width \maxwidth. This means
    % that they will get their normal width if they fit onto the page, but
    % are scaled down if they would overflow the margins.
    \makeatletter
    \def\maxwidth{\ifdim\Gin@nat@width>\linewidth\linewidth
    \else\Gin@nat@width\fi}
    \makeatother
    \let\Oldincludegraphics\includegraphics
    % Set max figure width to be 80% of text width, for now hardcoded.
    \renewcommand{\includegraphics}[1]{\Oldincludegraphics[width=.8\maxwidth]{#1}}
    % Ensure that by default, figures have no caption (until we provide a
    % proper Figure object with a Caption API and a way to capture that
    % in the conversion process - todo).
    \usepackage{caption}
    \DeclareCaptionLabelFormat{nolabel}{}
    \captionsetup{labelformat=nolabel}

    \usepackage{adjustbox} % Used to constrain images to a maximum size 
    \usepackage{xcolor} % Allow colors to be defined
    \usepackage{enumerate} % Needed for markdown enumerations to work
    \usepackage{geometry} % Used to adjust the document margins
    \usepackage{amsmath} % Equations
    \usepackage{amssymb} % Equations
    \usepackage{textcomp} % defines textquotesingle
    % Hack from http://tex.stackexchange.com/a/47451/13684:
    \AtBeginDocument{%
        \def\PYZsq{\textquotesingle}% Upright quotes in Pygmentized code
    }
    \usepackage{upquote} % Upright quotes for verbatim code
    \usepackage{eurosym} % defines \euro
    \usepackage[mathletters]{ucs} % Extended unicode (utf-8) support
    \usepackage[utf8x]{inputenc} % Allow utf-8 characters in the tex document
    \usepackage{fancyvrb} % verbatim replacement that allows latex
    \usepackage{grffile} % extends the file name processing of package graphics 
                         % to support a larger range 
    % The hyperref package gives us a pdf with properly built
    % internal navigation ('pdf bookmarks' for the table of contents,
    % internal cross-reference links, web links for URLs, etc.)
    \usepackage{hyperref}
    \usepackage{longtable} % longtable support required by pandoc >1.10
    \usepackage{booktabs}  % table support for pandoc > 1.12.2
    \usepackage[inline]{enumitem} % IRkernel/repr support (it uses the enumerate* environment)
    \usepackage[normalem]{ulem} % ulem is needed to support strikethroughs (\sout)
                                % normalem makes italics be italics, not underlines
    

    
    
    % Colors for the hyperref package
    \definecolor{urlcolor}{rgb}{0,.145,.698}
    \definecolor{linkcolor}{rgb}{.71,0.21,0.01}
    \definecolor{citecolor}{rgb}{.12,.54,.11}

    % ANSI colors
    \definecolor{ansi-black}{HTML}{3E424D}
    \definecolor{ansi-black-intense}{HTML}{282C36}
    \definecolor{ansi-red}{HTML}{E75C58}
    \definecolor{ansi-red-intense}{HTML}{B22B31}
    \definecolor{ansi-green}{HTML}{00A250}
    \definecolor{ansi-green-intense}{HTML}{007427}
    \definecolor{ansi-yellow}{HTML}{DDB62B}
    \definecolor{ansi-yellow-intense}{HTML}{B27D12}
    \definecolor{ansi-blue}{HTML}{208FFB}
    \definecolor{ansi-blue-intense}{HTML}{0065CA}
    \definecolor{ansi-magenta}{HTML}{D160C4}
    \definecolor{ansi-magenta-intense}{HTML}{A03196}
    \definecolor{ansi-cyan}{HTML}{60C6C8}
    \definecolor{ansi-cyan-intense}{HTML}{258F8F}
    \definecolor{ansi-white}{HTML}{C5C1B4}
    \definecolor{ansi-white-intense}{HTML}{A1A6B2}

    % commands and environments needed by pandoc snippets
    % extracted from the output of `pandoc -s`
    \providecommand{\tightlist}{%
      \setlength{\itemsep}{0pt}\setlength{\parskip}{0pt}}
    \DefineVerbatimEnvironment{Highlighting}{Verbatim}{commandchars=\\\{\}}
    % Add ',fontsize=\small' for more characters per line
    \newenvironment{Shaded}{}{}
    \newcommand{\KeywordTok}[1]{\textcolor[rgb]{0.00,0.44,0.13}{\textbf{{#1}}}}
    \newcommand{\DataTypeTok}[1]{\textcolor[rgb]{0.56,0.13,0.00}{{#1}}}
    \newcommand{\DecValTok}[1]{\textcolor[rgb]{0.25,0.63,0.44}{{#1}}}
    \newcommand{\BaseNTok}[1]{\textcolor[rgb]{0.25,0.63,0.44}{{#1}}}
    \newcommand{\FloatTok}[1]{\textcolor[rgb]{0.25,0.63,0.44}{{#1}}}
    \newcommand{\CharTok}[1]{\textcolor[rgb]{0.25,0.44,0.63}{{#1}}}
    \newcommand{\StringTok}[1]{\textcolor[rgb]{0.25,0.44,0.63}{{#1}}}
    \newcommand{\CommentTok}[1]{\textcolor[rgb]{0.38,0.63,0.69}{\textit{{#1}}}}
    \newcommand{\OtherTok}[1]{\textcolor[rgb]{0.00,0.44,0.13}{{#1}}}
    \newcommand{\AlertTok}[1]{\textcolor[rgb]{1.00,0.00,0.00}{\textbf{{#1}}}}
    \newcommand{\FunctionTok}[1]{\textcolor[rgb]{0.02,0.16,0.49}{{#1}}}
    \newcommand{\RegionMarkerTok}[1]{{#1}}
    \newcommand{\ErrorTok}[1]{\textcolor[rgb]{1.00,0.00,0.00}{\textbf{{#1}}}}
    \newcommand{\NormalTok}[1]{{#1}}
    
    % Additional commands for more recent versions of Pandoc
    \newcommand{\ConstantTok}[1]{\textcolor[rgb]{0.53,0.00,0.00}{{#1}}}
    \newcommand{\SpecialCharTok}[1]{\textcolor[rgb]{0.25,0.44,0.63}{{#1}}}
    \newcommand{\VerbatimStringTok}[1]{\textcolor[rgb]{0.25,0.44,0.63}{{#1}}}
    \newcommand{\SpecialStringTok}[1]{\textcolor[rgb]{0.73,0.40,0.53}{{#1}}}
    \newcommand{\ImportTok}[1]{{#1}}
    \newcommand{\DocumentationTok}[1]{\textcolor[rgb]{0.73,0.13,0.13}{\textit{{#1}}}}
    \newcommand{\AnnotationTok}[1]{\textcolor[rgb]{0.38,0.63,0.69}{\textbf{\textit{{#1}}}}}
    \newcommand{\CommentVarTok}[1]{\textcolor[rgb]{0.38,0.63,0.69}{\textbf{\textit{{#1}}}}}
    \newcommand{\VariableTok}[1]{\textcolor[rgb]{0.10,0.09,0.49}{{#1}}}
    \newcommand{\ControlFlowTok}[1]{\textcolor[rgb]{0.00,0.44,0.13}{\textbf{{#1}}}}
    \newcommand{\OperatorTok}[1]{\textcolor[rgb]{0.40,0.40,0.40}{{#1}}}
    \newcommand{\BuiltInTok}[1]{{#1}}
    \newcommand{\ExtensionTok}[1]{{#1}}
    \newcommand{\PreprocessorTok}[1]{\textcolor[rgb]{0.74,0.48,0.00}{{#1}}}
    \newcommand{\AttributeTok}[1]{\textcolor[rgb]{0.49,0.56,0.16}{{#1}}}
    \newcommand{\InformationTok}[1]{\textcolor[rgb]{0.38,0.63,0.69}{\textbf{\textit{{#1}}}}}
    \newcommand{\WarningTok}[1]{\textcolor[rgb]{0.38,0.63,0.69}{\textbf{\textit{{#1}}}}}
    
    
    % Define a nice break command that doesn't care if a line doesn't already
    % exist.
    \def\br{\hspace*{\fill} \\* }
    % Math Jax compatability definitions
    \def\gt{>}
    \def\lt{<}
    % Document parameters
    \title{matplotlib\_1}
    
    
    

    % Pygments definitions
    
\makeatletter
\def\PY@reset{\let\PY@it=\relax \let\PY@bf=\relax%
    \let\PY@ul=\relax \let\PY@tc=\relax%
    \let\PY@bc=\relax \let\PY@ff=\relax}
\def\PY@tok#1{\csname PY@tok@#1\endcsname}
\def\PY@toks#1+{\ifx\relax#1\empty\else%
    \PY@tok{#1}\expandafter\PY@toks\fi}
\def\PY@do#1{\PY@bc{\PY@tc{\PY@ul{%
    \PY@it{\PY@bf{\PY@ff{#1}}}}}}}
\def\PY#1#2{\PY@reset\PY@toks#1+\relax+\PY@do{#2}}

\expandafter\def\csname PY@tok@w\endcsname{\def\PY@tc##1{\textcolor[rgb]{0.73,0.73,0.73}{##1}}}
\expandafter\def\csname PY@tok@c\endcsname{\let\PY@it=\textit\def\PY@tc##1{\textcolor[rgb]{0.25,0.50,0.50}{##1}}}
\expandafter\def\csname PY@tok@cp\endcsname{\def\PY@tc##1{\textcolor[rgb]{0.74,0.48,0.00}{##1}}}
\expandafter\def\csname PY@tok@k\endcsname{\let\PY@bf=\textbf\def\PY@tc##1{\textcolor[rgb]{0.00,0.50,0.00}{##1}}}
\expandafter\def\csname PY@tok@kp\endcsname{\def\PY@tc##1{\textcolor[rgb]{0.00,0.50,0.00}{##1}}}
\expandafter\def\csname PY@tok@kt\endcsname{\def\PY@tc##1{\textcolor[rgb]{0.69,0.00,0.25}{##1}}}
\expandafter\def\csname PY@tok@o\endcsname{\def\PY@tc##1{\textcolor[rgb]{0.40,0.40,0.40}{##1}}}
\expandafter\def\csname PY@tok@ow\endcsname{\let\PY@bf=\textbf\def\PY@tc##1{\textcolor[rgb]{0.67,0.13,1.00}{##1}}}
\expandafter\def\csname PY@tok@nb\endcsname{\def\PY@tc##1{\textcolor[rgb]{0.00,0.50,0.00}{##1}}}
\expandafter\def\csname PY@tok@nf\endcsname{\def\PY@tc##1{\textcolor[rgb]{0.00,0.00,1.00}{##1}}}
\expandafter\def\csname PY@tok@nc\endcsname{\let\PY@bf=\textbf\def\PY@tc##1{\textcolor[rgb]{0.00,0.00,1.00}{##1}}}
\expandafter\def\csname PY@tok@nn\endcsname{\let\PY@bf=\textbf\def\PY@tc##1{\textcolor[rgb]{0.00,0.00,1.00}{##1}}}
\expandafter\def\csname PY@tok@ne\endcsname{\let\PY@bf=\textbf\def\PY@tc##1{\textcolor[rgb]{0.82,0.25,0.23}{##1}}}
\expandafter\def\csname PY@tok@nv\endcsname{\def\PY@tc##1{\textcolor[rgb]{0.10,0.09,0.49}{##1}}}
\expandafter\def\csname PY@tok@no\endcsname{\def\PY@tc##1{\textcolor[rgb]{0.53,0.00,0.00}{##1}}}
\expandafter\def\csname PY@tok@nl\endcsname{\def\PY@tc##1{\textcolor[rgb]{0.63,0.63,0.00}{##1}}}
\expandafter\def\csname PY@tok@ni\endcsname{\let\PY@bf=\textbf\def\PY@tc##1{\textcolor[rgb]{0.60,0.60,0.60}{##1}}}
\expandafter\def\csname PY@tok@na\endcsname{\def\PY@tc##1{\textcolor[rgb]{0.49,0.56,0.16}{##1}}}
\expandafter\def\csname PY@tok@nt\endcsname{\let\PY@bf=\textbf\def\PY@tc##1{\textcolor[rgb]{0.00,0.50,0.00}{##1}}}
\expandafter\def\csname PY@tok@nd\endcsname{\def\PY@tc##1{\textcolor[rgb]{0.67,0.13,1.00}{##1}}}
\expandafter\def\csname PY@tok@s\endcsname{\def\PY@tc##1{\textcolor[rgb]{0.73,0.13,0.13}{##1}}}
\expandafter\def\csname PY@tok@sd\endcsname{\let\PY@it=\textit\def\PY@tc##1{\textcolor[rgb]{0.73,0.13,0.13}{##1}}}
\expandafter\def\csname PY@tok@si\endcsname{\let\PY@bf=\textbf\def\PY@tc##1{\textcolor[rgb]{0.73,0.40,0.53}{##1}}}
\expandafter\def\csname PY@tok@se\endcsname{\let\PY@bf=\textbf\def\PY@tc##1{\textcolor[rgb]{0.73,0.40,0.13}{##1}}}
\expandafter\def\csname PY@tok@sr\endcsname{\def\PY@tc##1{\textcolor[rgb]{0.73,0.40,0.53}{##1}}}
\expandafter\def\csname PY@tok@ss\endcsname{\def\PY@tc##1{\textcolor[rgb]{0.10,0.09,0.49}{##1}}}
\expandafter\def\csname PY@tok@sx\endcsname{\def\PY@tc##1{\textcolor[rgb]{0.00,0.50,0.00}{##1}}}
\expandafter\def\csname PY@tok@m\endcsname{\def\PY@tc##1{\textcolor[rgb]{0.40,0.40,0.40}{##1}}}
\expandafter\def\csname PY@tok@gh\endcsname{\let\PY@bf=\textbf\def\PY@tc##1{\textcolor[rgb]{0.00,0.00,0.50}{##1}}}
\expandafter\def\csname PY@tok@gu\endcsname{\let\PY@bf=\textbf\def\PY@tc##1{\textcolor[rgb]{0.50,0.00,0.50}{##1}}}
\expandafter\def\csname PY@tok@gd\endcsname{\def\PY@tc##1{\textcolor[rgb]{0.63,0.00,0.00}{##1}}}
\expandafter\def\csname PY@tok@gi\endcsname{\def\PY@tc##1{\textcolor[rgb]{0.00,0.63,0.00}{##1}}}
\expandafter\def\csname PY@tok@gr\endcsname{\def\PY@tc##1{\textcolor[rgb]{1.00,0.00,0.00}{##1}}}
\expandafter\def\csname PY@tok@ge\endcsname{\let\PY@it=\textit}
\expandafter\def\csname PY@tok@gs\endcsname{\let\PY@bf=\textbf}
\expandafter\def\csname PY@tok@gp\endcsname{\let\PY@bf=\textbf\def\PY@tc##1{\textcolor[rgb]{0.00,0.00,0.50}{##1}}}
\expandafter\def\csname PY@tok@go\endcsname{\def\PY@tc##1{\textcolor[rgb]{0.53,0.53,0.53}{##1}}}
\expandafter\def\csname PY@tok@gt\endcsname{\def\PY@tc##1{\textcolor[rgb]{0.00,0.27,0.87}{##1}}}
\expandafter\def\csname PY@tok@err\endcsname{\def\PY@bc##1{\setlength{\fboxsep}{0pt}\fcolorbox[rgb]{1.00,0.00,0.00}{1,1,1}{\strut ##1}}}
\expandafter\def\csname PY@tok@kc\endcsname{\let\PY@bf=\textbf\def\PY@tc##1{\textcolor[rgb]{0.00,0.50,0.00}{##1}}}
\expandafter\def\csname PY@tok@kd\endcsname{\let\PY@bf=\textbf\def\PY@tc##1{\textcolor[rgb]{0.00,0.50,0.00}{##1}}}
\expandafter\def\csname PY@tok@kn\endcsname{\let\PY@bf=\textbf\def\PY@tc##1{\textcolor[rgb]{0.00,0.50,0.00}{##1}}}
\expandafter\def\csname PY@tok@kr\endcsname{\let\PY@bf=\textbf\def\PY@tc##1{\textcolor[rgb]{0.00,0.50,0.00}{##1}}}
\expandafter\def\csname PY@tok@bp\endcsname{\def\PY@tc##1{\textcolor[rgb]{0.00,0.50,0.00}{##1}}}
\expandafter\def\csname PY@tok@fm\endcsname{\def\PY@tc##1{\textcolor[rgb]{0.00,0.00,1.00}{##1}}}
\expandafter\def\csname PY@tok@vc\endcsname{\def\PY@tc##1{\textcolor[rgb]{0.10,0.09,0.49}{##1}}}
\expandafter\def\csname PY@tok@vg\endcsname{\def\PY@tc##1{\textcolor[rgb]{0.10,0.09,0.49}{##1}}}
\expandafter\def\csname PY@tok@vi\endcsname{\def\PY@tc##1{\textcolor[rgb]{0.10,0.09,0.49}{##1}}}
\expandafter\def\csname PY@tok@vm\endcsname{\def\PY@tc##1{\textcolor[rgb]{0.10,0.09,0.49}{##1}}}
\expandafter\def\csname PY@tok@sa\endcsname{\def\PY@tc##1{\textcolor[rgb]{0.73,0.13,0.13}{##1}}}
\expandafter\def\csname PY@tok@sb\endcsname{\def\PY@tc##1{\textcolor[rgb]{0.73,0.13,0.13}{##1}}}
\expandafter\def\csname PY@tok@sc\endcsname{\def\PY@tc##1{\textcolor[rgb]{0.73,0.13,0.13}{##1}}}
\expandafter\def\csname PY@tok@dl\endcsname{\def\PY@tc##1{\textcolor[rgb]{0.73,0.13,0.13}{##1}}}
\expandafter\def\csname PY@tok@s2\endcsname{\def\PY@tc##1{\textcolor[rgb]{0.73,0.13,0.13}{##1}}}
\expandafter\def\csname PY@tok@sh\endcsname{\def\PY@tc##1{\textcolor[rgb]{0.73,0.13,0.13}{##1}}}
\expandafter\def\csname PY@tok@s1\endcsname{\def\PY@tc##1{\textcolor[rgb]{0.73,0.13,0.13}{##1}}}
\expandafter\def\csname PY@tok@mb\endcsname{\def\PY@tc##1{\textcolor[rgb]{0.40,0.40,0.40}{##1}}}
\expandafter\def\csname PY@tok@mf\endcsname{\def\PY@tc##1{\textcolor[rgb]{0.40,0.40,0.40}{##1}}}
\expandafter\def\csname PY@tok@mh\endcsname{\def\PY@tc##1{\textcolor[rgb]{0.40,0.40,0.40}{##1}}}
\expandafter\def\csname PY@tok@mi\endcsname{\def\PY@tc##1{\textcolor[rgb]{0.40,0.40,0.40}{##1}}}
\expandafter\def\csname PY@tok@il\endcsname{\def\PY@tc##1{\textcolor[rgb]{0.40,0.40,0.40}{##1}}}
\expandafter\def\csname PY@tok@mo\endcsname{\def\PY@tc##1{\textcolor[rgb]{0.40,0.40,0.40}{##1}}}
\expandafter\def\csname PY@tok@ch\endcsname{\let\PY@it=\textit\def\PY@tc##1{\textcolor[rgb]{0.25,0.50,0.50}{##1}}}
\expandafter\def\csname PY@tok@cm\endcsname{\let\PY@it=\textit\def\PY@tc##1{\textcolor[rgb]{0.25,0.50,0.50}{##1}}}
\expandafter\def\csname PY@tok@cpf\endcsname{\let\PY@it=\textit\def\PY@tc##1{\textcolor[rgb]{0.25,0.50,0.50}{##1}}}
\expandafter\def\csname PY@tok@c1\endcsname{\let\PY@it=\textit\def\PY@tc##1{\textcolor[rgb]{0.25,0.50,0.50}{##1}}}
\expandafter\def\csname PY@tok@cs\endcsname{\let\PY@it=\textit\def\PY@tc##1{\textcolor[rgb]{0.25,0.50,0.50}{##1}}}

\def\PYZbs{\char`\\}
\def\PYZus{\char`\_}
\def\PYZob{\char`\{}
\def\PYZcb{\char`\}}
\def\PYZca{\char`\^}
\def\PYZam{\char`\&}
\def\PYZlt{\char`\<}
\def\PYZgt{\char`\>}
\def\PYZsh{\char`\#}
\def\PYZpc{\char`\%}
\def\PYZdl{\char`\$}
\def\PYZhy{\char`\-}
\def\PYZsq{\char`\'}
\def\PYZdq{\char`\"}
\def\PYZti{\char`\~}
% for compatibility with earlier versions
\def\PYZat{@}
\def\PYZlb{[}
\def\PYZrb{]}
\makeatother


    % Exact colors from NB
    \definecolor{incolor}{rgb}{0.0, 0.0, 0.5}
    \definecolor{outcolor}{rgb}{0.545, 0.0, 0.0}



    
    % Prevent overflowing lines due to hard-to-break entities
    \sloppy 
    % Setup hyperref package
    \hypersetup{
      breaklinks=true,  % so long urls are correctly broken across lines
      colorlinks=true,
      urlcolor=urlcolor,
      linkcolor=linkcolor,
      citecolor=citecolor,
      }
    % Slightly bigger margins than the latex defaults
    
    \geometry{verbose,tmargin=1in,bmargin=1in,lmargin=1in,rmargin=1in}
    
    

    \begin{document}
    
    
    \maketitle
    
    

    
    \hypertarget{plotting-1}{%
\section{Plotting 1}\label{plotting-1}}

We have a handle on python now: we understand the data structures and
enough about working with them to move on to stuff more directly
relevant to data analysis. We know how to get data into Pandas from
files, how to manipulate DataFrames and how to do basic statistics.

Let's get started on making figures, arguably the best way to convey
information about our data.

    \hypertarget{the-packages}{%
\subsubsection{The packages}\label{the-packages}}

    \begin{Verbatim}[commandchars=\\\{\}]
{\color{incolor}In [{\color{incolor}1}]:} \PY{k+kn}{import} \PY{n+nn}{pandas} \PY{k}{as} \PY{n+nn}{pd}     \PY{c+c1}{\PYZsh{}load the pandas package and call it pd}
        \PY{k+kn}{import} \PY{n+nn}{matplotlib}\PY{n+nn}{.}\PY{n+nn}{pyplot} \PY{k}{as} \PY{n+nn}{plt}   \PY{c+c1}{\PYZsh{} load the pyplot set of tools from the package matplotlib. Name it plt for short.}
        
        \PY{c+c1}{\PYZsh{} This following is a jupyter magic command. It tells jupyter to insert the plots into the notebook}
        \PY{c+c1}{\PYZsh{} rather than a new window.}
        \PY{o}{\PYZpc{}}\PY{k}{matplotlib} inline      
\end{Verbatim}


    matplotlib is a very popular package that bundles tools for creating
visualizations. The documentation is
\href{https://matplotlib.org/contents.html}{here}. We will look at some
specific plot types in class, but you can learn about many different
types \href{https://matplotlib.org/gallery/index.html}{thumbnail
gallery}. {[}Warning: not all the figures in the thumbnail gallery are
good figures.{]}

Copy the \texttt{gdp\_components\_simple.csv} file into your cwd (or
load it using a file path to its location) and load it into pandas.

    \begin{Verbatim}[commandchars=\\\{\}]
{\color{incolor}In [{\color{incolor}2}]:} \PY{n}{gdp} \PY{o}{=} \PY{n}{pd}\PY{o}{.}\PY{n}{read\PYZus{}csv}\PY{p}{(}\PY{l+s+s1}{\PYZsq{}}\PY{l+s+s1}{gdp\PYZus{}components\PYZus{}simple.csv}\PY{l+s+s1}{\PYZsq{}}\PY{p}{,} \PY{n}{index\PYZus{}col}\PY{o}{=}\PY{l+m+mi}{0}\PY{p}{)}  \PY{c+c1}{\PYZsh{} load data from file, make date the index}
        
        \PY{n+nb}{print}\PY{p}{(}\PY{n}{gdp}\PY{o}{.}\PY{n}{head}\PY{p}{(}\PY{l+m+mi}{2}\PY{p}{)}\PY{p}{)}                                    \PY{c+c1}{\PYZsh{} print the first and last few rows to make sure all is well}
        \PY{n+nb}{print}\PY{p}{(}\PY{l+s+s1}{\PYZsq{}}\PY{l+s+se}{\PYZbs{}n}\PY{l+s+s1}{\PYZsq{}}\PY{p}{,} \PY{n}{gdp}\PY{o}{.}\PY{n}{tail}\PY{p}{(}\PY{l+m+mi}{2}\PY{p}{)}\PY{p}{)}
\end{Verbatim}


    \begin{Verbatim}[commandchars=\\\{\}]
         GDPA   GPDIA    GCEA  EXPGSA  IMPGSA
DATE                                         
1929  104.556  17.170   9.622   5.939   5.556
1930   92.160  11.428  10.273   4.444   4.121

            GDPA     GPDIA      GCEA    EXPGSA    IMPGSA
DATE                                                   
2016  18707.189  3169.887  3290.979  2217.576  2738.146
2017  19485.394  3367.965  3374.444  2350.175  2928.596

    \end{Verbatim}

    I don't like these variable names.

    \begin{Verbatim}[commandchars=\\\{\}]
{\color{incolor}In [{\color{incolor}3}]:} \PY{n}{gdp}\PY{o}{.}\PY{n}{rename}\PY{p}{(}\PY{n}{columns} \PY{o}{=} \PY{p}{\PYZob{}}\PY{l+s+s1}{\PYZsq{}}\PY{l+s+s1}{GDPA}\PY{l+s+s1}{\PYZsq{}}\PY{p}{:}\PY{l+s+s1}{\PYZsq{}}\PY{l+s+s1}{gdp}\PY{l+s+s1}{\PYZsq{}}\PY{p}{,} \PY{l+s+s1}{\PYZsq{}}\PY{l+s+s1}{GPDIA}\PY{l+s+s1}{\PYZsq{}}\PY{p}{:}\PY{l+s+s1}{\PYZsq{}}\PY{l+s+s1}{inv}\PY{l+s+s1}{\PYZsq{}}\PY{p}{,} \PY{l+s+s1}{\PYZsq{}}\PY{l+s+s1}{GCEA}\PY{l+s+s1}{\PYZsq{}}\PY{p}{:}\PY{l+s+s1}{\PYZsq{}}\PY{l+s+s1}{gov}\PY{l+s+s1}{\PYZsq{}}\PY{p}{,} \PY{l+s+s1}{\PYZsq{}}\PY{l+s+s1}{EXPGSA}\PY{l+s+s1}{\PYZsq{}}\PY{p}{:}\PY{l+s+s1}{\PYZsq{}}\PY{l+s+s1}{ex}\PY{l+s+s1}{\PYZsq{}}\PY{p}{,} \PY{l+s+s1}{\PYZsq{}}\PY{l+s+s1}{IMPGSA}\PY{l+s+s1}{\PYZsq{}}\PY{p}{:}\PY{l+s+s1}{\PYZsq{}}\PY{l+s+s1}{im}\PY{l+s+s1}{\PYZsq{}} \PY{p}{\PYZcb{}}\PY{p}{,} \PY{n}{inplace}\PY{o}{=}\PY{k+kc}{True}\PY{p}{)}
        \PY{n}{gdp}
\end{Verbatim}


\begin{Verbatim}[commandchars=\\\{\}]
{\color{outcolor}Out[{\color{outcolor}3}]:}             gdp       inv       gov        ex        im
        DATE                                                   
        1929    104.556    17.170     9.622     5.939     5.556
        1930     92.160    11.428    10.273     4.444     4.121
        1931     77.391     6.549    10.169     2.906     2.905
        1932     59.522     1.819     8.946     1.975     1.932
        1933     57.154     2.276     8.875     1.987     1.929
        1934     66.800     4.296    10.721     2.561     2.239
        1935     74.241     7.370    11.151     2.769     2.982
        1936     84.830     9.391    13.398     3.007     3.154
        1937     93.003    12.967    13.119     4.039     3.961
        1938     87.352     7.944    14.170     3.811     2.845
        1939     93.437    10.229    15.165     3.969     3.136
        1940    102.899    14.579    15.562     4.897     3.426
        1941    129.309    19.369    27.836     5.482     4.449
        1942    165.952    11.762    65.440     4.375     4.627
        1943    203.084     7.405    98.023     4.034     6.280
        1944    224.447     9.180   108.643     4.880     6.904
        1945    228.007    12.400    96.396     6.781     7.547
        1946    227.535    33.128    43.027    14.156     6.974
        1947    249.616    37.131    39.827    18.740     7.933
        1948    274.468    50.347    43.755    15.547    10.060
        1949    272.475    39.099    49.808    14.484     9.249
        1950    299.827    56.530    50.515    12.350    11.612
        1951    346.914    62.759    73.304    17.099    14.586
        1952    367.341    57.274    89.583    16.459    15.295
        1953    389.218    60.414    96.765    15.313    16.014
        1954    390.549    58.070    92.467    15.836    15.432
        1955    425.478    73.754    92.958    17.677    17.199
        1956    449.353    77.685    98.180    21.284    18.923
        1957    474.039    76.505   107.163    24.017    19.942
        1958    481.229    70.947   114.138    20.560    20.022
        {\ldots}         {\ldots}       {\ldots}       {\ldots}       {\ldots}       {\ldots}
        1988   5236.438   936.963  1078.855   444.601   553.993
        1989   5641.580   999.701  1151.862   504.289   591.031
        1990   5963.144   993.448  1238.556   551.873   629.727
        1991   6158.129   944.344  1298.951   594.931   623.544
        1992   6520.327  1013.006  1344.500   633.053   667.791
        1993   6858.559  1106.826  1364.922   654.799   719.973
        1994   7287.236  1256.484  1402.274   720.937   813.424
        1995   7639.749  1317.489  1449.431   812.810   902.572
        1996   8073.122  1432.055  1492.848   867.589   963.966
        1997   8577.552  1595.600  1547.133   953.803  1055.774
        1998   9062.817  1736.671  1611.609   952.979  1115.690
        1999   9630.663  1887.059  1720.360   992.778  1248.612
        2000  10252.347  2038.408  1826.845  1096.255  1471.305
        2001  10581.822  1934.842  1949.275  1024.636  1392.565
        2002  10936.418  1930.417  2088.717   998.741  1424.143
        2003  11458.246  2027.056  2211.208  1036.177  1539.304
        2004  12213.730  2281.253  2338.889  1177.631  1796.706
        2005  13036.637  2534.720  2475.992  1305.225  2026.418
        2006  13814.609  2700.954  2624.234  1472.613  2243.538
        2007  14451.860  2673.011  2790.844  1660.853  2379.280
        2008  14712.845  2477.613  2981.990  1837.055  2560.143
        2009  14448.932  1929.664  3073.512  1581.996  1978.447
        2010  14992.052  2165.473  3154.647  1846.280  2360.183
        2011  15542.582  2332.562  3148.372  2102.995  2682.456
        2012  16197.007  2621.754  3137.010  2191.280  2759.851
        2013  16784.851  2826.013  3132.409  2273.428  2764.210
        2014  17521.747  3038.931  3167.041  2371.027  2879.284
        2015  18219.297  3211.971  3234.210  2265.047  2786.461
        2016  18707.189  3169.887  3290.979  2217.576  2738.146
        2017  19485.394  3367.965  3374.444  2350.175  2928.596
        
        [89 rows x 5 columns]
\end{Verbatim}
            
    Let's get plotting. matplotlib graphics are based around two new object
types. 1. The figure object: think of this as the canvas we will draw
figures onto 2. The axes object: think of this as the figure itself and
all the components

To create a new figure, we call the \texttt{subplots()} method of
\texttt{plt}. Notice the use of multiple assignment.

    \begin{Verbatim}[commandchars=\\\{\}]
{\color{incolor}In [{\color{incolor}4}]:} \PY{n}{fig}\PY{p}{,} \PY{n}{ax} \PY{o}{=} \PY{n}{plt}\PY{o}{.}\PY{n}{subplots}\PY{p}{(}\PY{p}{)}    \PY{c+c1}{\PYZsh{} passing no arguments gets us one fig object and one axes object}
\end{Verbatim}


    \begin{center}
    \adjustimage{max size={0.9\linewidth}{0.9\paperheight}}{output_8_0.png}
    \end{center}
    { \hspace*{\fill} \\}
    
    \begin{Verbatim}[commandchars=\\\{\}]
{\color{incolor}In [{\color{incolor}5}]:} \PY{n+nb}{print}\PY{p}{(}\PY{n+nb}{type}\PY{p}{(}\PY{n}{fig}\PY{p}{)}\PY{p}{)}
        
        \PY{n+nb}{print}\PY{p}{(}\PY{n+nb}{type}\PY{p}{(}\PY{n}{ax}\PY{p}{)}\PY{p}{)}
\end{Verbatim}


    \begin{Verbatim}[commandchars=\\\{\}]
<class 'matplotlib.figure.Figure'>
<class 'matplotlib.axes.\_subplots.AxesSubplot'>

    \end{Verbatim}

    We apply methods to the axes to actually plot the data. Here is a
scatter plot. {[}Try \texttt{ax.} and hit TAB\ldots{}{]}

    \begin{Verbatim}[commandchars=\\\{\}]
{\color{incolor}In [{\color{incolor}6}]:} \PY{n}{fig}\PY{p}{,} \PY{n}{ax} \PY{o}{=} \PY{n}{plt}\PY{o}{.}\PY{n}{subplots}\PY{p}{(}\PY{p}{)} 
        \PY{n}{ax}\PY{o}{.}\PY{n}{plot}\PY{p}{(}\PY{n}{gdp}\PY{o}{.}\PY{n}{index}\PY{p}{,} \PY{n}{gdp}\PY{p}{[}\PY{l+s+s1}{\PYZsq{}}\PY{l+s+s1}{gdp}\PY{l+s+s1}{\PYZsq{}}\PY{p}{]}\PY{p}{)}                  \PY{c+c1}{\PYZsh{} scatter plot of gdp vs. time}
\end{Verbatim}


\begin{Verbatim}[commandchars=\\\{\}]
{\color{outcolor}Out[{\color{outcolor}6}]:} [<matplotlib.lines.Line2D at 0x116c6dda0>]
\end{Verbatim}
            
    \begin{center}
    \adjustimage{max size={0.9\linewidth}{0.9\paperheight}}{output_11_1.png}
    \end{center}
    { \hspace*{\fill} \\}
    
    First, note that the plot is a Line2D object. This is absolutely not
important for us, but when you see jupyter print out
\texttt{\textless{}matplotlib.lines.Line2D\ at\ ...\textgreater{}} that
is what it is telling us. Everything in python is an object.

Second, a scatter plot needs two columns of data, one for the
x-coordinate and one for the y-coordinate. I am using \texttt{gdp} for
the y-coordinate and the years for the x-coordinate. I set years as the
index variable, so to retrieve it I used the \texttt{.index} attribute.

Third, this plot needs some work. I do not like this line color. More
importantly, I am missing labels and a title. These are
\textbf{extremely important.}

    \begin{Verbatim}[commandchars=\\\{\}]
{\color{incolor}In [{\color{incolor}7}]:} \PY{n}{fig}\PY{p}{,} \PY{n}{ax} \PY{o}{=} \PY{n}{plt}\PY{o}{.}\PY{n}{subplots}\PY{p}{(}\PY{p}{)} 
        \PY{n}{ax}\PY{o}{.}\PY{n}{plot}\PY{p}{(}\PY{n}{gdp}\PY{o}{.}\PY{n}{index}\PY{p}{,} \PY{n}{gdp}\PY{p}{[}\PY{l+s+s1}{\PYZsq{}}\PY{l+s+s1}{gdp}\PY{l+s+s1}{\PYZsq{}}\PY{p}{]}\PY{p}{,}        \PY{c+c1}{\PYZsh{} line plot of gdp vs. time}
                \PY{n}{color}\PY{o}{=}\PY{l+s+s1}{\PYZsq{}}\PY{l+s+s1}{red}\PY{l+s+s1}{\PYZsq{}}                   \PY{c+c1}{\PYZsh{} set the line color to red}
               \PY{p}{)}                  
        
        \PY{n}{ax}\PY{o}{.}\PY{n}{set\PYZus{}ylabel}\PY{p}{(}\PY{l+s+s1}{\PYZsq{}}\PY{l+s+s1}{billions of dollars}\PY{l+s+s1}{\PYZsq{}}\PY{p}{)}  \PY{c+c1}{\PYZsh{} add the y\PYZhy{}axis label}
        \PY{n}{ax}\PY{o}{.}\PY{n}{set\PYZus{}xlabel}\PY{p}{(}\PY{l+s+s1}{\PYZsq{}}\PY{l+s+s1}{year}\PY{l+s+s1}{\PYZsq{}}\PY{p}{)}                 \PY{c+c1}{\PYZsh{} add the x\PYZhy{}axis label}
        \PY{n}{ax}\PY{o}{.}\PY{n}{set\PYZus{}title}\PY{p}{(}\PY{l+s+s1}{\PYZsq{}}\PY{l+s+s1}{U.S. Gross Domestic Product}\PY{l+s+s1}{\PYZsq{}}\PY{p}{)}
\end{Verbatim}


\begin{Verbatim}[commandchars=\\\{\}]
{\color{outcolor}Out[{\color{outcolor}7}]:} Text(0.5,1,'U.S. Gross Domestic Product')
\end{Verbatim}
            
    \begin{center}
    \adjustimage{max size={0.9\linewidth}{0.9\paperheight}}{output_13_1.png}
    \end{center}
    { \hspace*{\fill} \\}
    
    This is looking pretty good. While I am a fanatic when it comes to
labeling things, I probably wouldn't label the x-axis. You have to have
some faith in the reader.

I also do not like `boxing' my plots. There is a philosophy about
visualizations that says: Every mark on your figure should convey
information. If it does not, then it is clutter and should be removed. I
am not sure who developed this philosophy
\href{https://en.wikipedia.org/wiki/Marie_Kondo}{(Marie Kondo?)} but I
think it is a useful benchmark.

    \begin{Verbatim}[commandchars=\\\{\}]
{\color{incolor}In [{\color{incolor}8}]:} \PY{n}{fig}\PY{p}{,} \PY{n}{ax} \PY{o}{=} \PY{n}{plt}\PY{o}{.}\PY{n}{subplots}\PY{p}{(}\PY{p}{)} 
        \PY{n}{ax}\PY{o}{.}\PY{n}{plot}\PY{p}{(}\PY{n}{gdp}\PY{o}{.}\PY{n}{index}\PY{p}{,} \PY{n}{gdp}\PY{p}{[}\PY{l+s+s1}{\PYZsq{}}\PY{l+s+s1}{gdp}\PY{l+s+s1}{\PYZsq{}}\PY{p}{]}\PY{p}{,}        \PY{c+c1}{\PYZsh{} line plot of gdp vs. time}
                \PY{n}{color}\PY{o}{=}\PY{l+s+s1}{\PYZsq{}}\PY{l+s+s1}{red}\PY{l+s+s1}{\PYZsq{}}                   \PY{c+c1}{\PYZsh{} set the line color to red}
               \PY{p}{)}  
        
        \PY{n}{ax}\PY{o}{.}\PY{n}{set\PYZus{}ylabel}\PY{p}{(}\PY{l+s+s1}{\PYZsq{}}\PY{l+s+s1}{billions of dollars}\PY{l+s+s1}{\PYZsq{}}\PY{p}{)}  \PY{c+c1}{\PYZsh{} add the y\PYZhy{}axis label}
        \PY{c+c1}{\PYZsh{} ax.set\PYZus{}xlabel(\PYZsq{}year\PYZsq{})                 \PYZsh{} add the x\PYZhy{}axis label}
        \PY{n}{ax}\PY{o}{.}\PY{n}{set\PYZus{}title}\PY{p}{(}\PY{l+s+s1}{\PYZsq{}}\PY{l+s+s1}{U.S. Gross Domestic Product}\PY{l+s+s1}{\PYZsq{}}\PY{p}{)}
        
        \PY{n}{ax}\PY{o}{.}\PY{n}{spines}\PY{p}{[}\PY{l+s+s1}{\PYZsq{}}\PY{l+s+s1}{right}\PY{l+s+s1}{\PYZsq{}}\PY{p}{]}\PY{o}{.}\PY{n}{set\PYZus{}visible}\PY{p}{(}\PY{k+kc}{False}\PY{p}{)} \PY{c+c1}{\PYZsh{} get ride of the line on the right}
        \PY{n}{ax}\PY{o}{.}\PY{n}{spines}\PY{p}{[}\PY{l+s+s1}{\PYZsq{}}\PY{l+s+s1}{top}\PY{l+s+s1}{\PYZsq{}}\PY{p}{]}\PY{o}{.}\PY{n}{set\PYZus{}visible}\PY{p}{(}\PY{k+kc}{False}\PY{p}{)}   \PY{c+c1}{\PYZsh{} get rid of the line on top}
\end{Verbatim}


    \begin{center}
    \adjustimage{max size={0.9\linewidth}{0.9\paperheight}}{output_15_0.png}
    \end{center}
    { \hspace*{\fill} \\}
    
    \hypertarget{practice-plots}{%
\subsubsection{Practice: Plots}\label{practice-plots}}

Take a few minutes and try the following. Feel free to chat with those
around if you get stuck. The TA and I are here, too.

\begin{enumerate}
\def\labelenumi{\arabic{enumi}.}
\tightlist
\item
  Copy the code from the last plot and add a second line that plots
  `gov'. To do this, just add a new line of code to the existing code.
  \texttt{ax.plot(gdp.index,\ gdp{[}\textquotesingle{}gov{]}{]})}
\end{enumerate}

    \begin{Verbatim}[commandchars=\\\{\}]
{\color{incolor}In [{\color{incolor}9}]:} \PY{n}{fig}\PY{p}{,} \PY{n}{ax} \PY{o}{=} \PY{n}{plt}\PY{o}{.}\PY{n}{subplots}\PY{p}{(}\PY{p}{)} 
        \PY{n}{ax}\PY{o}{.}\PY{n}{plot}\PY{p}{(}\PY{n}{gdp}\PY{o}{.}\PY{n}{index}\PY{p}{,} \PY{n}{gdp}\PY{p}{[}\PY{l+s+s1}{\PYZsq{}}\PY{l+s+s1}{gdp}\PY{l+s+s1}{\PYZsq{}}\PY{p}{]}\PY{p}{,}        \PY{c+c1}{\PYZsh{} line plot of gdp vs. time}
                \PY{n}{color}\PY{o}{=}\PY{l+s+s1}{\PYZsq{}}\PY{l+s+s1}{red}\PY{l+s+s1}{\PYZsq{}}                   \PY{c+c1}{\PYZsh{} set the line color to red}
               \PY{p}{)}  
        
        \PY{n}{ax}\PY{o}{.}\PY{n}{plot}\PY{p}{(}\PY{n}{gdp}\PY{o}{.}\PY{n}{index}\PY{p}{,} \PY{n}{gdp}\PY{p}{[}\PY{l+s+s2}{\PYZdq{}}\PY{l+s+s2}{gov}\PY{l+s+s2}{\PYZdq{}}\PY{p}{]}\PY{p}{,} 
               \PY{n}{color} \PY{o}{=} \PY{l+s+s2}{\PYZdq{}}\PY{l+s+s2}{blue}\PY{l+s+s2}{\PYZdq{}}\PY{p}{,} 
               \PY{n}{alpha} \PY{o}{=} \PY{o}{.}\PY{l+m+mi}{5}\PY{p}{,} 
               \PY{n}{linestyle} \PY{o}{=} \PY{l+s+s2}{\PYZdq{}}\PY{l+s+s2}{\PYZhy{}\PYZhy{}}\PY{l+s+s2}{\PYZdq{}}\PY{p}{)}
        
        \PY{n}{ax}\PY{o}{.}\PY{n}{set\PYZus{}ylabel}\PY{p}{(}\PY{l+s+s1}{\PYZsq{}}\PY{l+s+s1}{billions of dollars}\PY{l+s+s1}{\PYZsq{}}\PY{p}{)}  \PY{c+c1}{\PYZsh{} add the y\PYZhy{}axis label}
        \PY{c+c1}{\PYZsh{} ax.set\PYZus{}xlabel(\PYZsq{}year\PYZsq{})                 \PYZsh{} add the x\PYZhy{}axis label}
        \PY{n}{ax}\PY{o}{.}\PY{n}{set\PYZus{}title}\PY{p}{(}\PY{l+s+s1}{\PYZsq{}}\PY{l+s+s1}{U.S. Gross Domestic Product and Government Spendings}\PY{l+s+s1}{\PYZsq{}}\PY{p}{)}
        
        \PY{n}{ax}\PY{o}{.}\PY{n}{spines}\PY{p}{[}\PY{l+s+s1}{\PYZsq{}}\PY{l+s+s1}{right}\PY{l+s+s1}{\PYZsq{}}\PY{p}{]}\PY{o}{.}\PY{n}{set\PYZus{}visible}\PY{p}{(}\PY{k+kc}{False}\PY{p}{)} \PY{c+c1}{\PYZsh{} get ride of the line on the right}
        \PY{n}{ax}\PY{o}{.}\PY{n}{spines}\PY{p}{[}\PY{l+s+s1}{\PYZsq{}}\PY{l+s+s1}{top}\PY{l+s+s1}{\PYZsq{}}\PY{p}{]}\PY{o}{.}\PY{n}{set\PYZus{}visible}\PY{p}{(}\PY{k+kc}{False}\PY{p}{)}   \PY{c+c1}{\PYZsh{} get rid of the line on top}
\end{Verbatim}


    \begin{center}
    \adjustimage{max size={0.9\linewidth}{0.9\paperheight}}{output_17_0.png}
    \end{center}
    { \hspace*{\fill} \\}
    
    \begin{enumerate}
\def\labelenumi{\arabic{enumi}.}
\setcounter{enumi}{1}
\tightlist
\item
  Modify your code to give the figure a better title
\item
  Modify your code to make government consumption blue
\item
  Modify your code to add the argument \texttt{alpha=0.5} to the plot
  method for gov. What does it change? If you want to learn more try
  `alpha composite' in Google.
\item
  Modify your code to make the gov line dashed. Try the argument
  \texttt{linestyle=\textquotesingle{}-\/-\textquotesingle{}}. What is
  linestyle `-.' or `:' ?
\end{enumerate}

    \hypertarget{a-few-more-options-to-get-us-started}{%
\subsubsection{A few more options to get us
started}\label{a-few-more-options-to-get-us-started}}

We have two lines on our figure. Which one is which? Not labeling our
line is malpractice. Two approaches

\begin{enumerate}
\def\labelenumi{\arabic{enumi}.}
\tightlist
\item
  Add a legend
\item
  Add text to the figure
\end{enumerate}

Both are good options. I prefer the second for simple plots.

    \begin{Verbatim}[commandchars=\\\{\}]
{\color{incolor}In [{\color{incolor}10}]:} \PY{c+c1}{\PYZsh{} The first option. Add labels to your plot commands, then call ax.legend.}
         
         \PY{n}{fig}\PY{p}{,} \PY{n}{ax} \PY{o}{=} \PY{n}{plt}\PY{o}{.}\PY{n}{subplots}\PY{p}{(}\PY{p}{)} 
         \PY{n}{ax}\PY{o}{.}\PY{n}{plot}\PY{p}{(}\PY{n}{gdp}\PY{o}{.}\PY{n}{index}\PY{p}{,} \PY{n}{gdp}\PY{p}{[}\PY{l+s+s1}{\PYZsq{}}\PY{l+s+s1}{gdp}\PY{l+s+s1}{\PYZsq{}}\PY{p}{]}\PY{p}{,}        \PY{c+c1}{\PYZsh{} line plot of gdp vs. time}
                 \PY{n}{color}\PY{o}{=}\PY{l+s+s1}{\PYZsq{}}\PY{l+s+s1}{red}\PY{l+s+s1}{\PYZsq{}}\PY{p}{,}                   \PY{c+c1}{\PYZsh{} set the line color to red}
                \PY{n}{label} \PY{o}{=} \PY{l+s+s1}{\PYZsq{}}\PY{l+s+s1}{GDP}\PY{l+s+s1}{\PYZsq{}}
                \PY{p}{)}  
         
         \PY{n}{ax}\PY{o}{.}\PY{n}{plot}\PY{p}{(}\PY{n}{gdp}\PY{o}{.}\PY{n}{index}\PY{p}{,} \PY{n}{gdp}\PY{p}{[}\PY{l+s+s1}{\PYZsq{}}\PY{l+s+s1}{gov}\PY{l+s+s1}{\PYZsq{}}\PY{p}{]}\PY{p}{,}        \PY{c+c1}{\PYZsh{} line plot of gdp vs. time}
                 \PY{n}{color}\PY{o}{=}\PY{l+s+s1}{\PYZsq{}}\PY{l+s+s1}{blue}\PY{l+s+s1}{\PYZsq{}}\PY{p}{,}                   \PY{c+c1}{\PYZsh{} set the line color to blue}
                 \PY{n}{alpha} \PY{o}{=} \PY{l+m+mf}{0.5}\PY{p}{,}
                 \PY{n}{linestyle} \PY{o}{=} \PY{l+s+s1}{\PYZsq{}}\PY{l+s+s1}{:}\PY{l+s+s1}{\PYZsq{}}\PY{p}{,}
                 \PY{n}{label} \PY{o}{=} \PY{l+s+s1}{\PYZsq{}}\PY{l+s+s1}{Gov. Spending}\PY{l+s+s1}{\PYZsq{}}
                \PY{p}{)}  
         \PY{n}{ax}\PY{o}{.}\PY{n}{set\PYZus{}ylabel}\PY{p}{(}\PY{l+s+s1}{\PYZsq{}}\PY{l+s+s1}{billions of dollars}\PY{l+s+s1}{\PYZsq{}}\PY{p}{)}  \PY{c+c1}{\PYZsh{} add the y\PYZhy{}axis label}
         \PY{c+c1}{\PYZsh{} ax.set\PYZus{}xlabel(\PYZsq{}year\PYZsq{})                 \PYZsh{} add the x\PYZhy{}axis label}
         \PY{n}{ax}\PY{o}{.}\PY{n}{set\PYZus{}title}\PY{p}{(}\PY{l+s+s1}{\PYZsq{}}\PY{l+s+s1}{U.S. Gross Domestic Product and Government Spending}\PY{l+s+s1}{\PYZsq{}}\PY{p}{)}
         
         \PY{n}{ax}\PY{o}{.}\PY{n}{spines}\PY{p}{[}\PY{l+s+s1}{\PYZsq{}}\PY{l+s+s1}{right}\PY{l+s+s1}{\PYZsq{}}\PY{p}{]}\PY{o}{.}\PY{n}{set\PYZus{}visible}\PY{p}{(}\PY{k+kc}{False}\PY{p}{)} \PY{c+c1}{\PYZsh{} get ride of the line on the right}
         \PY{n}{ax}\PY{o}{.}\PY{n}{spines}\PY{p}{[}\PY{l+s+s1}{\PYZsq{}}\PY{l+s+s1}{top}\PY{l+s+s1}{\PYZsq{}}\PY{p}{]}\PY{o}{.}\PY{n}{set\PYZus{}visible}\PY{p}{(}\PY{k+kc}{False}\PY{p}{)}   \PY{c+c1}{\PYZsh{} get rid of the line on top}
         
         \PY{n}{ax}\PY{o}{.}\PY{n}{legend}\PY{p}{(}\PY{n}{frameon}\PY{o}{=}\PY{k+kc}{False}\PY{p}{)}                           \PY{c+c1}{\PYZsh{} Show the legend. frameon=False kills the box around the legend}
\end{Verbatim}


\begin{Verbatim}[commandchars=\\\{\}]
{\color{outcolor}Out[{\color{outcolor}10}]:} <matplotlib.legend.Legend at 0x117048898>
\end{Verbatim}
            
    \begin{center}
    \adjustimage{max size={0.9\linewidth}{0.9\paperheight}}{output_20_1.png}
    \end{center}
    { \hspace*{\fill} \\}
    
    Ah, I feel much better now that I know which line is which. Here is the
second approach.

    \begin{Verbatim}[commandchars=\\\{\}]
{\color{incolor}In [{\color{incolor}11}]:} \PY{c+c1}{\PYZsh{} The second option. Add text using the annotate method. Note that I can leave the labels in the plot commands.}
         
         \PY{n}{fig}\PY{p}{,} \PY{n}{ax} \PY{o}{=} \PY{n}{plt}\PY{o}{.}\PY{n}{subplots}\PY{p}{(}\PY{p}{)} 
         \PY{n}{ax}\PY{o}{.}\PY{n}{plot}\PY{p}{(}\PY{n}{gdp}\PY{o}{.}\PY{n}{index}\PY{p}{,} \PY{n}{gdp}\PY{p}{[}\PY{l+s+s1}{\PYZsq{}}\PY{l+s+s1}{gdp}\PY{l+s+s1}{\PYZsq{}}\PY{p}{]}\PY{p}{,}        \PY{c+c1}{\PYZsh{} line plot of gdp vs. time}
                 \PY{n}{color}\PY{o}{=}\PY{l+s+s1}{\PYZsq{}}\PY{l+s+s1}{red}\PY{l+s+s1}{\PYZsq{}}\PY{p}{,}                   \PY{c+c1}{\PYZsh{} set the line color to red}
                \PY{n}{label} \PY{o}{=} \PY{l+s+s1}{\PYZsq{}}\PY{l+s+s1}{GDP}\PY{l+s+s1}{\PYZsq{}}
                \PY{p}{)}  
         
         \PY{n}{ax}\PY{o}{.}\PY{n}{plot}\PY{p}{(}\PY{n}{gdp}\PY{o}{.}\PY{n}{index}\PY{p}{,} \PY{n}{gdp}\PY{p}{[}\PY{l+s+s1}{\PYZsq{}}\PY{l+s+s1}{gov}\PY{l+s+s1}{\PYZsq{}}\PY{p}{]}\PY{p}{,}        \PY{c+c1}{\PYZsh{} line plot of gdp vs. time}
                 \PY{n}{color}\PY{o}{=}\PY{l+s+s1}{\PYZsq{}}\PY{l+s+s1}{blue}\PY{l+s+s1}{\PYZsq{}}\PY{p}{,}                   \PY{c+c1}{\PYZsh{} set the line color to blue}
                 \PY{n}{alpha} \PY{o}{=} \PY{l+m+mf}{0.5}\PY{p}{,}
                 \PY{n}{linestyle} \PY{o}{=} \PY{l+s+s1}{\PYZsq{}}\PY{l+s+s1}{:}\PY{l+s+s1}{\PYZsq{}}\PY{p}{,}
                 \PY{n}{label} \PY{o}{=} \PY{l+s+s1}{\PYZsq{}}\PY{l+s+s1}{Gov. Spending}\PY{l+s+s1}{\PYZsq{}}
                \PY{p}{)}  
         \PY{n}{ax}\PY{o}{.}\PY{n}{set\PYZus{}ylabel}\PY{p}{(}\PY{l+s+s1}{\PYZsq{}}\PY{l+s+s1}{billions of dollars}\PY{l+s+s1}{\PYZsq{}}\PY{p}{)}  \PY{c+c1}{\PYZsh{} add the y\PYZhy{}axis label}
         \PY{c+c1}{\PYZsh{} ax.set\PYZus{}xlabel(\PYZsq{}year\PYZsq{})                 \PYZsh{} add the x\PYZhy{}axis label}
         \PY{n}{ax}\PY{o}{.}\PY{n}{set\PYZus{}title}\PY{p}{(}\PY{l+s+s1}{\PYZsq{}}\PY{l+s+s1}{U.S. Gross Domestic Product and Government Spending}\PY{l+s+s1}{\PYZsq{}}\PY{p}{)}
         
         \PY{n}{ax}\PY{o}{.}\PY{n}{spines}\PY{p}{[}\PY{l+s+s1}{\PYZsq{}}\PY{l+s+s1}{right}\PY{l+s+s1}{\PYZsq{}}\PY{p}{]}\PY{o}{.}\PY{n}{set\PYZus{}visible}\PY{p}{(}\PY{k+kc}{False}\PY{p}{)} \PY{c+c1}{\PYZsh{} get ride of the line on the right}
         \PY{n}{ax}\PY{o}{.}\PY{n}{spines}\PY{p}{[}\PY{l+s+s1}{\PYZsq{}}\PY{l+s+s1}{top}\PY{l+s+s1}{\PYZsq{}}\PY{p}{]}\PY{o}{.}\PY{n}{set\PYZus{}visible}\PY{p}{(}\PY{k+kc}{False}\PY{p}{)}   \PY{c+c1}{\PYZsh{} get rid of the line on top}
         
         \PY{n}{ax}\PY{o}{.}\PY{n}{text}\PY{p}{(}\PY{l+m+mi}{1989}\PY{p}{,} \PY{l+m+mi}{8500}\PY{p}{,} \PY{l+s+s1}{\PYZsq{}}\PY{l+s+s1}{GDP}\PY{l+s+s1}{\PYZsq{}}\PY{p}{)}            \PY{c+c1}{\PYZsh{} text(x, y, string)}
         \PY{n}{ax}\PY{o}{.}\PY{n}{text}\PY{p}{(}\PY{l+m+mi}{1999}\PY{p}{,} \PY{l+m+mi}{4500}\PY{p}{,} \PY{l+s+s1}{\PYZsq{}}\PY{l+s+s1}{Gov. Spending}\PY{l+s+s1}{\PYZsq{}}\PY{p}{)}            \PY{c+c1}{\PYZsh{} text(x, y, string)}
\end{Verbatim}


\begin{Verbatim}[commandchars=\\\{\}]
{\color{outcolor}Out[{\color{outcolor}11}]:} Text(1999,4500,'Gov. Spending')
\end{Verbatim}
            
    \begin{center}
    \adjustimage{max size={0.9\linewidth}{0.9\paperheight}}{output_22_1.png}
    \end{center}
    { \hspace*{\fill} \\}
    
    \hypertarget{about-graphical-excellence}{%
\subsubsection{About graphical
excellence}\label{about-graphical-excellence}}

\href{https://www.edwardtufte.com/tufte/books_vdqi}{Edward Tufte's}
\emph{The Visual Display of Quantitative Information} is a masterpiece
of `thinking hard about visualizations.' The book is worth a read (or a
look through). His Principles of Graphic Excellence are great to keep in
mind. Two of my favorites: * Graphical excellence consists of complex
ideas communicated with clarity, precision, and efficiency * Graphical
excellence is that which gives the viewer the greatest number of ideas
in the shortest time with the least ink in the smallest space.

Let us all strive for graphical excellence.

This listicle \emph{Ten Simple Rules for Better Figures}
(\href{https://journals.plos.org/ploscompbiol/article?id=10.1371/journal.pcbi.1003833\#s10}{Rougier,
Droettboom, and Bourne, 2014}) has some good advice, too. My favorite is
\textbf{Do not trust the defaults.}

\href{http://media.juiceanalytics.com/images/blog/excel_line_graph.png}{Not
long, ago, this was Excel's default chart formatting}

    \hypertarget{getting-plots-out-of-your-notebook}{%
\subsubsection{Getting plots out of your
notebook}\label{getting-plots-out-of-your-notebook}}

While I love jupyter notebooks, my research output is usually an article
distributed as a pdf.

    \begin{Verbatim}[commandchars=\\\{\}]
{\color{incolor}In [{\color{incolor}12}]:} \PY{n}{fig}\PY{p}{,} \PY{n}{ax} \PY{o}{=} \PY{n}{plt}\PY{o}{.}\PY{n}{subplots}\PY{p}{(}\PY{p}{)} 
         \PY{n}{ax}\PY{o}{.}\PY{n}{plot}\PY{p}{(}\PY{n}{gdp}\PY{o}{.}\PY{n}{index}\PY{p}{,} \PY{n}{gdp}\PY{p}{[}\PY{l+s+s1}{\PYZsq{}}\PY{l+s+s1}{gdp}\PY{l+s+s1}{\PYZsq{}}\PY{p}{]}\PY{p}{,}        \PY{c+c1}{\PYZsh{} line plot of gdp vs. time}
                 \PY{n}{color}\PY{o}{=}\PY{l+s+s1}{\PYZsq{}}\PY{l+s+s1}{red}\PY{l+s+s1}{\PYZsq{}}\PY{p}{,}                   \PY{c+c1}{\PYZsh{} set the line color to red}
                \PY{n}{label} \PY{o}{=} \PY{l+s+s1}{\PYZsq{}}\PY{l+s+s1}{GDP}\PY{l+s+s1}{\PYZsq{}}
                \PY{p}{)}  
         
         \PY{n}{ax}\PY{o}{.}\PY{n}{plot}\PY{p}{(}\PY{n}{gdp}\PY{o}{.}\PY{n}{index}\PY{p}{,} \PY{n}{gdp}\PY{p}{[}\PY{l+s+s1}{\PYZsq{}}\PY{l+s+s1}{gov}\PY{l+s+s1}{\PYZsq{}}\PY{p}{]}\PY{p}{,}        \PY{c+c1}{\PYZsh{} line plot of gdp vs. time}
                 \PY{n}{color}\PY{o}{=}\PY{l+s+s1}{\PYZsq{}}\PY{l+s+s1}{blue}\PY{l+s+s1}{\PYZsq{}}\PY{p}{,}                   \PY{c+c1}{\PYZsh{} set the line color to blue}
                 \PY{n}{alpha} \PY{o}{=} \PY{l+m+mf}{0.5}\PY{p}{,}
                 \PY{n}{linestyle} \PY{o}{=} \PY{l+s+s1}{\PYZsq{}}\PY{l+s+s1}{:}\PY{l+s+s1}{\PYZsq{}}\PY{p}{,}
                 \PY{n}{label} \PY{o}{=} \PY{l+s+s1}{\PYZsq{}}\PY{l+s+s1}{Gov. Spending}\PY{l+s+s1}{\PYZsq{}}
                \PY{p}{)}  
         \PY{n}{ax}\PY{o}{.}\PY{n}{set\PYZus{}ylabel}\PY{p}{(}\PY{l+s+s1}{\PYZsq{}}\PY{l+s+s1}{billions of dollars}\PY{l+s+s1}{\PYZsq{}}\PY{p}{)}  \PY{c+c1}{\PYZsh{} add the y\PYZhy{}axis label}
         \PY{c+c1}{\PYZsh{} ax.set\PYZus{}xlabel(\PYZsq{}year\PYZsq{})                 \PYZsh{} add the x\PYZhy{}axis label}
         \PY{n}{ax}\PY{o}{.}\PY{n}{set\PYZus{}title}\PY{p}{(}\PY{l+s+s1}{\PYZsq{}}\PY{l+s+s1}{U.S. Gross Domestic Product and Government Spending}\PY{l+s+s1}{\PYZsq{}}\PY{p}{)}
         
         \PY{n}{ax}\PY{o}{.}\PY{n}{spines}\PY{p}{[}\PY{l+s+s1}{\PYZsq{}}\PY{l+s+s1}{right}\PY{l+s+s1}{\PYZsq{}}\PY{p}{]}\PY{o}{.}\PY{n}{set\PYZus{}visible}\PY{p}{(}\PY{k+kc}{False}\PY{p}{)} \PY{c+c1}{\PYZsh{} get ride of the line on the right}
         \PY{n}{ax}\PY{o}{.}\PY{n}{spines}\PY{p}{[}\PY{l+s+s1}{\PYZsq{}}\PY{l+s+s1}{top}\PY{l+s+s1}{\PYZsq{}}\PY{p}{]}\PY{o}{.}\PY{n}{set\PYZus{}visible}\PY{p}{(}\PY{k+kc}{False}\PY{p}{)}   \PY{c+c1}{\PYZsh{} get rid of the line on top}
         
         \PY{n}{ax}\PY{o}{.}\PY{n}{text}\PY{p}{(}\PY{l+m+mi}{1989}\PY{p}{,} \PY{l+m+mi}{8500}\PY{p}{,} \PY{l+s+s1}{\PYZsq{}}\PY{l+s+s1}{GDP}\PY{l+s+s1}{\PYZsq{}}\PY{p}{)}            \PY{c+c1}{\PYZsh{} text(x, y, string)}
         \PY{n}{ax}\PY{o}{.}\PY{n}{text}\PY{p}{(}\PY{l+m+mi}{1999}\PY{p}{,} \PY{l+m+mi}{4500}\PY{p}{,} \PY{l+s+s1}{\PYZsq{}}\PY{l+s+s1}{Gov. Spending}\PY{l+s+s1}{\PYZsq{}}\PY{p}{)}            \PY{c+c1}{\PYZsh{} text(x, y, string)}
         
         \PY{n}{plt}\PY{o}{.}\PY{n}{savefig}\PY{p}{(}\PY{l+s+s1}{\PYZsq{}}\PY{l+s+s1}{gdp.pdf}\PY{l+s+s1}{\PYZsq{}}\PY{p}{,} \PY{n}{bbox\PYZus{}inches}\PY{o}{=}\PY{l+s+s1}{\PYZsq{}}\PY{l+s+s1}{tight}\PY{l+s+s1}{\PYZsq{}}\PY{p}{)}          \PY{c+c1}{\PYZsh{} Create a pdf and save to cwd }
         \PY{n}{plt}\PY{o}{.}\PY{n}{savefig}\PY{p}{(}\PY{l+s+s1}{\PYZsq{}}\PY{l+s+s1}{../gdp.png}\PY{l+s+s1}{\PYZsq{}}\PY{p}{)}          \PY{c+c1}{\PYZsh{} Create a png and save to the folder that contains the cwd}
\end{Verbatim}


    \begin{center}
    \adjustimage{max size={0.9\linewidth}{0.9\paperheight}}{output_25_0.png}
    \end{center}
    { \hspace*{\fill} \\}
    
    When saving a pdf, I use the
\texttt{bbox\_inches=\textquotesingle{}tight\textquotesingle{}} argument
to kill extra whitespace around the figure. You can also set things like
orientation, dpi, and metadata. Check the documentation if you need to
tweak your output.

    \hypertarget{more-plot-types}{%
\subsubsection{More plot types}\label{more-plot-types}}

The line plot is the tip of the iceberg. matplotlib support many plot
types. Let's take a look at histograms.

How variable is US gdp growth?

    \begin{Verbatim}[commandchars=\\\{\}]
{\color{incolor}In [{\color{incolor}13}]:} \PY{c+c1}{\PYZsh{} Create a histogram of gdp growth rates.}
         
         \PY{n}{gdp}\PY{p}{[}\PY{l+s+s1}{\PYZsq{}}\PY{l+s+s1}{gdp\PYZus{}growth}\PY{l+s+s1}{\PYZsq{}}\PY{p}{]} \PY{o}{=} \PY{n}{gdp}\PY{p}{[}\PY{l+s+s1}{\PYZsq{}}\PY{l+s+s1}{gdp}\PY{l+s+s1}{\PYZsq{}}\PY{p}{]}\PY{o}{.}\PY{n}{pct\PYZus{}change}\PY{p}{(}\PY{p}{)}\PY{o}{*}\PY{l+m+mi}{100} \PY{c+c1}{\PYZsh{} pct\PYZus{}change() creates growth rates NOT percent change. Not a self\PYZhy{}documenting name.}
         \PY{n}{gdp}\PY{o}{.}\PY{n}{head}\PY{p}{(}\PY{p}{)}
\end{Verbatim}


\begin{Verbatim}[commandchars=\\\{\}]
{\color{outcolor}Out[{\color{outcolor}13}]:}           gdp     inv     gov     ex     im  gdp\_growth
         DATE                                                   
         1929  104.556  17.170   9.622  5.939  5.556         NaN
         1930   92.160  11.428  10.273  4.444  4.121  -11.855848
         1931   77.391   6.549  10.169  2.906  2.905  -16.025391
         1932   59.522   1.819   8.946  1.975  1.932  -23.089248
         1933   57.154   2.276   8.875  1.987  1.929   -3.978361
\end{Verbatim}
            
    We could have used the \texttt{diff()} or the \texttt{shift()} methods
to do something similar, but wow, pct\_change is so luxe. A quick plot
to take a look.

    \begin{Verbatim}[commandchars=\\\{\}]
{\color{incolor}In [{\color{incolor}14}]:} \PY{n}{fig}\PY{p}{,} \PY{n}{ax} \PY{o}{=} \PY{n}{plt}\PY{o}{.}\PY{n}{subplots}\PY{p}{(}\PY{p}{)} 
         \PY{n}{ax}\PY{o}{.}\PY{n}{plot}\PY{p}{(}\PY{n}{gdp}\PY{o}{.}\PY{n}{index}\PY{p}{,} \PY{n}{gdp}\PY{p}{[}\PY{l+s+s1}{\PYZsq{}}\PY{l+s+s1}{gdp\PYZus{}growth}\PY{l+s+s1}{\PYZsq{}}\PY{p}{]}\PY{p}{,}        \PY{c+c1}{\PYZsh{} line plot of gdp vs. time}
                 \PY{n}{color}\PY{o}{=}\PY{l+s+s1}{\PYZsq{}}\PY{l+s+s1}{red}\PY{l+s+s1}{\PYZsq{}}\PY{p}{,}                   \PY{c+c1}{\PYZsh{} set the line color to red}
                \PY{n}{label} \PY{o}{=} \PY{l+s+s1}{\PYZsq{}}\PY{l+s+s1}{GDP Growth}\PY{l+s+s1}{\PYZsq{}}
                \PY{p}{)}  
         
         \PY{n}{ax}\PY{o}{.}\PY{n}{set\PYZus{}ylabel}\PY{p}{(}\PY{l+s+s1}{\PYZsq{}}\PY{l+s+s1}{percent growth}\PY{l+s+s1}{\PYZsq{}}\PY{p}{)}  \PY{c+c1}{\PYZsh{} add the y\PYZhy{}axis label}
         \PY{n}{ax}\PY{o}{.}\PY{n}{set\PYZus{}title}\PY{p}{(}\PY{l+s+s1}{\PYZsq{}}\PY{l+s+s1}{U.S. Gross Domestic Product Growth Rates}\PY{l+s+s1}{\PYZsq{}}\PY{p}{)}
         
         \PY{n}{ax}\PY{o}{.}\PY{n}{spines}\PY{p}{[}\PY{l+s+s1}{\PYZsq{}}\PY{l+s+s1}{right}\PY{l+s+s1}{\PYZsq{}}\PY{p}{]}\PY{o}{.}\PY{n}{set\PYZus{}visible}\PY{p}{(}\PY{k+kc}{False}\PY{p}{)} \PY{c+c1}{\PYZsh{} get ride of the line on the right}
         \PY{n}{ax}\PY{o}{.}\PY{n}{spines}\PY{p}{[}\PY{l+s+s1}{\PYZsq{}}\PY{l+s+s1}{top}\PY{l+s+s1}{\PYZsq{}}\PY{p}{]}\PY{o}{.}\PY{n}{set\PYZus{}visible}\PY{p}{(}\PY{k+kc}{False}\PY{p}{)}   \PY{c+c1}{\PYZsh{} get rid of the line on top}
         
         \PY{n}{ax}\PY{o}{.}\PY{n}{axhline}\PY{p}{(}\PY{n}{y}\PY{o}{=}\PY{l+m+mi}{0}\PY{p}{,} \PY{n}{color}\PY{o}{=}\PY{l+s+s1}{\PYZsq{}}\PY{l+s+s1}{black}\PY{l+s+s1}{\PYZsq{}}\PY{p}{,} \PY{n}{linewidth}\PY{o}{=}\PY{l+m+mf}{0.75}\PY{p}{)}  \PY{c+c1}{\PYZsh{} Add a horizontal line at y=0}
\end{Verbatim}


\begin{Verbatim}[commandchars=\\\{\}]
{\color{outcolor}Out[{\color{outcolor}14}]:} <matplotlib.lines.Line2D at 0x116c6de10>
\end{Verbatim}
            
    \begin{center}
    \adjustimage{max size={0.9\linewidth}{0.9\paperheight}}{output_30_1.png}
    \end{center}
    { \hspace*{\fill} \\}
    
    The great depression and the WWII buildup really stick out.

Notice that I added a line at zero. My thinking is that this line adds
information: the reader can easily see that growth rates are mostly
positive and that the great depression was really bad.

It is also obvous that the volitility of gdp has fallen over time, but
let's approach a bit differently.

    \begin{Verbatim}[commandchars=\\\{\}]
{\color{incolor}In [{\color{incolor}15}]:} \PY{n}{fig}\PY{p}{,} \PY{n}{ax} \PY{o}{=} \PY{n}{plt}\PY{o}{.}\PY{n}{subplots}\PY{p}{(}\PY{p}{)} 
         
         \PY{c+c1}{\PYZsh{} hist does not like NaN. (I\PYZsq{}m a bit surprised.) I use the dropna() method to kill off the missing value}
         \PY{n}{ax}\PY{o}{.}\PY{n}{hist}\PY{p}{(}\PY{n}{gdp}\PY{p}{[}\PY{l+s+s1}{\PYZsq{}}\PY{l+s+s1}{gdp\PYZus{}growth}\PY{l+s+s1}{\PYZsq{}}\PY{p}{]}\PY{o}{.}\PY{n}{dropna}\PY{p}{(}\PY{p}{)}\PY{p}{,} \PY{n}{bins}\PY{o}{=}\PY{l+m+mi}{20}\PY{p}{,} \PY{n}{color}\PY{o}{=}\PY{l+s+s1}{\PYZsq{}}\PY{l+s+s1}{red}\PY{l+s+s1}{\PYZsq{}}\PY{p}{,} \PY{n}{alpha}\PY{o}{=}\PY{l+m+mf}{0.75}\PY{p}{)}        \PY{c+c1}{\PYZsh{} histogram of GDP growth rates}
               
         
         \PY{n}{ax}\PY{o}{.}\PY{n}{set\PYZus{}ylabel}\PY{p}{(}\PY{l+s+s1}{\PYZsq{}}\PY{l+s+s1}{Frequency}\PY{l+s+s1}{\PYZsq{}}\PY{p}{)}  \PY{c+c1}{\PYZsh{} add the y\PYZhy{}axis label}
         \PY{n}{ax}\PY{o}{.}\PY{n}{set\PYZus{}xlabel}\PY{p}{(}\PY{l+s+s1}{\PYZsq{}}\PY{l+s+s1}{Annual growth rate (}\PY{l+s+s1}{\PYZpc{}}\PY{l+s+s1}{)}\PY{l+s+s1}{\PYZsq{}}\PY{p}{)}
         \PY{n}{ax}\PY{o}{.}\PY{n}{set\PYZus{}title}\PY{p}{(}\PY{l+s+s1}{\PYZsq{}}\PY{l+s+s1}{Frequency of US GDP growth rates (1929\PYZhy{}2017)}\PY{l+s+s1}{\PYZsq{}}\PY{p}{)}
         
         \PY{n}{ax}\PY{o}{.}\PY{n}{spines}\PY{p}{[}\PY{l+s+s1}{\PYZsq{}}\PY{l+s+s1}{right}\PY{l+s+s1}{\PYZsq{}}\PY{p}{]}\PY{o}{.}\PY{n}{set\PYZus{}visible}\PY{p}{(}\PY{k+kc}{False}\PY{p}{)} \PY{c+c1}{\PYZsh{} get ride of the line on the right}
         \PY{n}{ax}\PY{o}{.}\PY{n}{spines}\PY{p}{[}\PY{l+s+s1}{\PYZsq{}}\PY{l+s+s1}{top}\PY{l+s+s1}{\PYZsq{}}\PY{p}{]}\PY{o}{.}\PY{n}{set\PYZus{}visible}\PY{p}{(}\PY{k+kc}{False}\PY{p}{)}   \PY{c+c1}{\PYZsh{} get rid of the line on top}
         
         \PY{c+c1}{\PYZsh{}ax.axhline(y=0, color=\PYZsq{}black\PYZsq{}, linewidth=0.75)  \PYZsh{} Add a horizontal line at y=0}
         
         \PY{n}{plt}\PY{o}{.}\PY{n}{show}\PY{p}{(}\PY{p}{)}
\end{Verbatim}


    \begin{center}
    \adjustimage{max size={0.9\linewidth}{0.9\paperheight}}{output_32_0.png}
    \end{center}
    { \hspace*{\fill} \\}
    
    \hypertarget{practice-histograms}{%
\subsubsection{Practice: Histograms}\label{practice-histograms}}

Take a few minutes and try the following. Feel free to chat with those
around if you get stuck. The TA and I are here, too.

\begin{enumerate}
\def\labelenumi{\arabic{enumi}.}
\tightlist
\item
  Break the data up into two periods: 1929-1985 and 1985-2017
\item
  Compute the mean and the standard deviation for the gdp growth rate in
  each sample.
\item
  Create a separate histogram for each sample. Make the early period
  historgram blue and the late historgram black. Make any changes to
  them that you deem appropriate.
\item
  Use text() to add the mean and std to a blank area of the histograms.
\item
  Save the two histograms as pdfs. Give them reasonable names.
\end{enumerate}

\emph{Challenging}. Can you find a way to store the value of the mean
and std to a variable and print the variable out on the histogram? Redo
part 4.

    \begin{Verbatim}[commandchars=\\\{\}]
{\color{incolor}In [{\color{incolor}58}]:} \PY{n}{gdp\PYZus{}one} \PY{o}{=} \PY{n}{gdp}\PY{o}{.}\PY{n}{loc}\PY{p}{[}\PY{l+m+mi}{1929}\PY{p}{:}\PY{l+m+mi}{1985}\PY{p}{,} \PY{p}{]}
         \PY{n}{gdp\PYZus{}two} \PY{o}{=} \PY{n}{gdp}\PY{o}{.}\PY{n}{loc}\PY{p}{[}\PY{l+m+mi}{1985}\PY{p}{:}\PY{l+m+mi}{2017}\PY{p}{,} \PY{p}{]}
         \PY{n}{gdp\PYZus{}one\PYZus{}mean} \PY{o}{=} \PY{n}{gdp\PYZus{}one}\PY{p}{[}\PY{l+s+s2}{\PYZdq{}}\PY{l+s+s2}{gdp\PYZus{}growth}\PY{l+s+s2}{\PYZdq{}}\PY{p}{]}\PY{o}{.}\PY{n}{mean}\PY{p}{(}\PY{p}{)}
         \PY{n}{gdp\PYZus{}two\PYZus{}mean} \PY{o}{=} \PY{n}{gdp\PYZus{}two}\PY{p}{[}\PY{l+s+s2}{\PYZdq{}}\PY{l+s+s2}{gdp\PYZus{}growth}\PY{l+s+s2}{\PYZdq{}}\PY{p}{]}\PY{o}{.}\PY{n}{mean}\PY{p}{(}\PY{p}{)}
         \PY{n}{gdp\PYZus{}one\PYZus{}std} \PY{o}{=} \PY{n}{pd}\PY{o}{.}\PY{n}{to\PYZus{}numeric}\PY{p}{(}\PY{n}{gdp\PYZus{}one}\PY{p}{[}\PY{l+s+s2}{\PYZdq{}}\PY{l+s+s2}{gdp\PYZus{}growth}\PY{l+s+s2}{\PYZdq{}}\PY{p}{]}\PY{o}{.}\PY{n}{std}\PY{p}{(}\PY{p}{)}\PY{p}{)}
         \PY{n}{gdp\PYZus{}two\PYZus{}std} \PY{o}{=} \PY{n}{pd}\PY{o}{.}\PY{n}{to\PYZus{}numeric}\PY{p}{(}\PY{n}{gdp\PYZus{}two}\PY{p}{[}\PY{l+s+s2}{\PYZdq{}}\PY{l+s+s2}{gdp\PYZus{}growth}\PY{l+s+s2}{\PYZdq{}}\PY{p}{]}\PY{o}{.}\PY{n}{std}\PY{p}{(}\PY{p}{)}\PY{p}{)}
         
         \PY{n}{fig}\PY{p}{,} \PY{n}{ax} \PY{o}{=} \PY{n}{plt}\PY{o}{.}\PY{n}{subplots}\PY{p}{(}\PY{p}{)}
         \PY{n}{ax}\PY{o}{.}\PY{n}{hist}\PY{p}{(}\PY{n}{gdp\PYZus{}one}\PY{p}{[}\PY{l+s+s2}{\PYZdq{}}\PY{l+s+s2}{gdp\PYZus{}growth}\PY{l+s+s2}{\PYZdq{}}\PY{p}{]}\PY{o}{.}\PY{n}{dropna}\PY{p}{(}\PY{p}{)}\PY{p}{,} \PY{n}{color} \PY{o}{=} \PY{l+s+s2}{\PYZdq{}}\PY{l+s+s2}{blue}\PY{l+s+s2}{\PYZdq{}}\PY{p}{)}
         \PY{n}{ax}\PY{o}{.}\PY{n}{set\PYZus{}title}\PY{p}{(}\PY{l+s+s2}{\PYZdq{}}\PY{l+s+s2}{U.S. GDP Growth Histogram, 1929\PYZhy{}1985}\PY{l+s+s2}{\PYZdq{}}\PY{p}{)}
         \PY{n}{ax}\PY{o}{.}\PY{n}{set\PYZus{}xlabel}\PY{p}{(}\PY{l+s+s2}{\PYZdq{}}\PY{l+s+s2}{GDP Growth, in Percentage}\PY{l+s+s2}{\PYZdq{}}\PY{p}{)}
         \PY{n}{ax}\PY{o}{.}\PY{n}{set\PYZus{}ylabel}\PY{p}{(}\PY{l+s+s2}{\PYZdq{}}\PY{l+s+s2}{Frequency}\PY{l+s+s2}{\PYZdq{}}\PY{p}{)}
         \PY{n}{plt}\PY{o}{.}\PY{n}{savefig}\PY{p}{(}\PY{l+s+s2}{\PYZdq{}}\PY{l+s+s2}{gdp\PYZus{}grwoth\PYZus{}hist\PYZus{}1929\PYZhy{}1985.pdf}\PY{l+s+s2}{\PYZdq{}}\PY{p}{,} \PY{n}{bbox\PYZus{}inches} \PY{o}{=} \PY{l+s+s2}{\PYZdq{}}\PY{l+s+s2}{tight}\PY{l+s+s2}{\PYZdq{}}\PY{p}{)}
         
         \PY{n}{fig2}\PY{p}{,} \PY{n}{ax2} \PY{o}{=} \PY{n}{plt}\PY{o}{.}\PY{n}{subplots}\PY{p}{(}\PY{p}{)}
         \PY{n}{ax2}\PY{o}{.}\PY{n}{hist}\PY{p}{(}\PY{n}{gdp\PYZus{}two}\PY{p}{[}\PY{l+s+s2}{\PYZdq{}}\PY{l+s+s2}{gdp\PYZus{}growth}\PY{l+s+s2}{\PYZdq{}}\PY{p}{]}\PY{o}{.}\PY{n}{dropna}\PY{p}{(}\PY{p}{)}\PY{p}{,} \PY{n}{color} \PY{o}{=} \PY{l+s+s2}{\PYZdq{}}\PY{l+s+s2}{black}\PY{l+s+s2}{\PYZdq{}}\PY{p}{)}
         \PY{n}{ax2}\PY{o}{.}\PY{n}{set\PYZus{}title}\PY{p}{(}\PY{l+s+s2}{\PYZdq{}}\PY{l+s+s2}{U.S. GDP Growth Histogram, 1985\PYZhy{}2017}\PY{l+s+s2}{\PYZdq{}}\PY{p}{)}
         \PY{n}{ax2}\PY{o}{.}\PY{n}{set\PYZus{}xlabel}\PY{p}{(}\PY{l+s+s2}{\PYZdq{}}\PY{l+s+s2}{GDP Growth, in Percentage}\PY{l+s+s2}{\PYZdq{}}\PY{p}{)}
         \PY{n}{ax2}\PY{o}{.}\PY{n}{set\PYZus{}ylabel}\PY{p}{(}\PY{l+s+s2}{\PYZdq{}}\PY{l+s+s2}{Frequency}\PY{l+s+s2}{\PYZdq{}}\PY{p}{)}
         \PY{n}{ax2}\PY{o}{.}\PY{n}{text}\PY{p}{(}\PY{n}{x} \PY{o}{=} \PY{o}{\PYZhy{}}\PY{l+m+mi}{1}\PY{p}{,} \PY{n}{y} \PY{o}{=} \PY{l+m+mi}{6}\PY{p}{,} \PY{n}{s} \PY{o}{=} \PY{l+s+s2}{\PYZdq{}}\PY{l+s+s2}{mean = }\PY{l+s+s2}{\PYZdq{}} \PY{o}{+} \PY{n+nb}{str}\PY{p}{(}\PY{n+nb}{round}\PY{p}{(}\PY{n}{gdp\PYZus{}two\PYZus{}mean}\PY{p}{,} \PY{n}{ndigits} \PY{o}{=} \PY{l+m+mi}{2}\PY{p}{)}\PY{p}{)}\PY{p}{,} \PY{n}{color} \PY{o}{=} \PY{l+s+s2}{\PYZdq{}}\PY{l+s+s2}{black}\PY{l+s+s2}{\PYZdq{}}\PY{p}{)}
         \PY{n}{ax2}\PY{o}{.}\PY{n}{text}\PY{p}{(}\PY{n}{x} \PY{o}{=} \PY{o}{\PYZhy{}}\PY{l+m+mi}{1}\PY{p}{,} \PY{n}{y} \PY{o}{=} \PY{l+m+mi}{5}\PY{p}{,} \PY{n}{s} \PY{o}{=} \PY{l+s+s2}{\PYZdq{}}\PY{l+s+s2}{standard dev. = }\PY{l+s+s2}{\PYZdq{}} \PY{o}{+} \PY{n+nb}{str}\PY{p}{(}\PY{n+nb}{round}\PY{p}{(}\PY{n}{gdp\PYZus{}two\PYZus{}std}\PY{p}{,} \PY{n}{ndigits} \PY{o}{=} \PY{l+m+mi}{2}\PY{p}{)}\PY{p}{)}\PY{p}{,} \PY{n}{color} \PY{o}{=} \PY{l+s+s2}{\PYZdq{}}\PY{l+s+s2}{black}\PY{l+s+s2}{\PYZdq{}}\PY{p}{)}
         \PY{n}{plt}\PY{o}{.}\PY{n}{savefig}\PY{p}{(}\PY{l+s+s2}{\PYZdq{}}\PY{l+s+s2}{gdp\PYZus{}growth\PYZus{}hist\PYZus{}1985\PYZhy{}1917.pdf}\PY{l+s+s2}{\PYZdq{}}\PY{p}{)}
\end{Verbatim}


    \begin{center}
    \adjustimage{max size={0.9\linewidth}{0.9\paperheight}}{output_34_0.png}
    \end{center}
    { \hspace*{\fill} \\}
    
    \begin{center}
    \adjustimage{max size={0.9\linewidth}{0.9\paperheight}}{output_34_1.png}
    \end{center}
    { \hspace*{\fill} \\}
    
    \hypertarget{subplots}{%
\subsubsection{Subplots}\label{subplots}}

We can generate several axes in one figure using the subplot() method.
{[}This method is not misnamed!{]}

    \begin{Verbatim}[commandchars=\\\{\}]
{\color{incolor}In [{\color{incolor}57}]:} \PY{n}{fig}\PY{p}{,} \PY{n}{ax} \PY{o}{=} \PY{n}{plt}\PY{o}{.}\PY{n}{subplots}\PY{p}{(}\PY{l+m+mi}{1}\PY{p}{,} \PY{l+m+mi}{2}\PY{p}{)}  \PY{c+c1}{\PYZsh{} one row, two columns of axes}
\end{Verbatim}


    \begin{center}
    \adjustimage{max size={0.9\linewidth}{0.9\paperheight}}{output_36_0.png}
    \end{center}
    { \hspace*{\fill} \\}
    
    \begin{Verbatim}[commandchars=\\\{\}]
{\color{incolor}In [{\color{incolor}58}]:} \PY{n+nb}{print}\PY{p}{(}\PY{n+nb}{type}\PY{p}{(}\PY{n}{ax}\PY{p}{)}\PY{p}{)}
\end{Verbatim}


    \begin{Verbatim}[commandchars=\\\{\}]
<class 'numpy.ndarray'>

    \end{Verbatim}

    So \texttt{ax} is now an array that holds the axes for each plot. Each
axes works just like before. Now we just have to tell python
\textbf{which} axes to act on.

    \begin{Verbatim}[commandchars=\\\{\}]
{\color{incolor}In [{\color{incolor}62}]:} \PY{c+c1}{\PYZsh{} Set a variable for plot color so I can change it everywhere easily}
         \PY{n}{my\PYZus{}plot\PYZus{}color} \PY{o}{=} \PY{l+s+s1}{\PYZsq{}}\PY{l+s+s1}{black}\PY{l+s+s1}{\PYZsq{}}
         
         \PY{c+c1}{\PYZsh{} I am using the figsize parameter here. It takes (width, height) in inches. }
         \PY{n}{fig}\PY{p}{,} \PY{n}{ax} \PY{o}{=} \PY{n}{plt}\PY{o}{.}\PY{n}{subplots}\PY{p}{(}\PY{l+m+mi}{1}\PY{p}{,} \PY{l+m+mi}{2}\PY{p}{,} \PY{n}{figsize}\PY{o}{=}\PY{p}{(}\PY{l+m+mi}{10}\PY{p}{,}\PY{l+m+mi}{4}\PY{p}{)}\PY{p}{)}  \PY{c+c1}{\PYZsh{} one row, two columns of axes}
         
         \PY{c+c1}{\PYZsh{} The fist plot}
         \PY{n}{ax}\PY{p}{[}\PY{l+m+mi}{0}\PY{p}{]}\PY{o}{.}\PY{n}{plot}\PY{p}{(}\PY{n}{gdp}\PY{o}{.}\PY{n}{index}\PY{p}{,} \PY{n}{gdp}\PY{p}{[}\PY{l+s+s1}{\PYZsq{}}\PY{l+s+s1}{gdp\PYZus{}growth}\PY{l+s+s1}{\PYZsq{}}\PY{p}{]}\PY{p}{,} \PY{n}{color}\PY{o}{=}\PY{n}{my\PYZus{}plot\PYZus{}color}\PY{p}{,} \PY{n}{label} \PY{o}{=} \PY{l+s+s1}{\PYZsq{}}\PY{l+s+s1}{GDP Growth}\PY{l+s+s1}{\PYZsq{}}\PY{p}{)}     \PY{c+c1}{\PYZsh{} a line plot of GDP growth rates}
         \PY{n}{ax}\PY{p}{[}\PY{l+m+mi}{0}\PY{p}{]}\PY{o}{.}\PY{n}{axhline}\PY{p}{(}\PY{n}{y}\PY{o}{=}\PY{l+m+mi}{0}\PY{p}{,} \PY{n}{color}\PY{o}{=}\PY{l+s+s1}{\PYZsq{}}\PY{l+s+s1}{black}\PY{l+s+s1}{\PYZsq{}}\PY{p}{,} \PY{n}{linewidth}\PY{o}{=}\PY{l+m+mf}{0.75}\PY{p}{)}  \PY{c+c1}{\PYZsh{} Add a horizontal line at y=0}
         \PY{n}{ax}\PY{p}{[}\PY{l+m+mi}{0}\PY{p}{]}\PY{o}{.}\PY{n}{set\PYZus{}xlabel}\PY{p}{(}\PY{l+s+s1}{\PYZsq{}}\PY{l+s+s1}{year}\PY{l+s+s1}{\PYZsq{}}\PY{p}{)}
         \PY{n}{ax}\PY{p}{[}\PY{l+m+mi}{0}\PY{p}{]}\PY{o}{.}\PY{n}{set\PYZus{}title}\PY{p}{(}\PY{l+s+s1}{\PYZsq{}}\PY{l+s+s1}{GDP growth rates}\PY{l+s+s1}{\PYZsq{}}\PY{p}{)}
         \PY{n}{ax}\PY{p}{[}\PY{l+m+mi}{0}\PY{p}{]}\PY{o}{.}\PY{n}{spines}\PY{p}{[}\PY{l+s+s1}{\PYZsq{}}\PY{l+s+s1}{right}\PY{l+s+s1}{\PYZsq{}}\PY{p}{]}\PY{o}{.}\PY{n}{set\PYZus{}visible}\PY{p}{(}\PY{k+kc}{False}\PY{p}{)} \PY{c+c1}{\PYZsh{} get ride of the line on the right}
         \PY{n}{ax}\PY{p}{[}\PY{l+m+mi}{0}\PY{p}{]}\PY{o}{.}\PY{n}{spines}\PY{p}{[}\PY{l+s+s1}{\PYZsq{}}\PY{l+s+s1}{top}\PY{l+s+s1}{\PYZsq{}}\PY{p}{]}\PY{o}{.}\PY{n}{set\PYZus{}visible}\PY{p}{(}\PY{k+kc}{False}\PY{p}{)}   \PY{c+c1}{\PYZsh{} get rid of the line on top}
         
         \PY{c+c1}{\PYZsh{} The second plot}
         \PY{n}{ax}\PY{p}{[}\PY{l+m+mi}{1}\PY{p}{]}\PY{o}{.}\PY{n}{hist}\PY{p}{(}\PY{n}{gdp}\PY{p}{[}\PY{l+s+s1}{\PYZsq{}}\PY{l+s+s1}{gdp\PYZus{}growth}\PY{l+s+s1}{\PYZsq{}}\PY{p}{]}\PY{o}{.}\PY{n}{dropna}\PY{p}{(}\PY{p}{)}\PY{p}{,} \PY{n}{bins}\PY{o}{=}\PY{l+m+mi}{20}\PY{p}{,} \PY{n}{color}\PY{o}{=}\PY{n}{my\PYZus{}plot\PYZus{}color}\PY{p}{,} \PY{n}{alpha}\PY{o}{=}\PY{l+m+mf}{0.25}\PY{p}{)}        \PY{c+c1}{\PYZsh{} histogram of GDP growth rates}
         \PY{n}{ax}\PY{p}{[}\PY{l+m+mi}{1}\PY{p}{]}\PY{o}{.}\PY{n}{set\PYZus{}xlabel}\PY{p}{(}\PY{l+s+s1}{\PYZsq{}}\PY{l+s+s1}{annual growth rate}\PY{l+s+s1}{\PYZsq{}}\PY{p}{)}
         \PY{n}{ax}\PY{p}{[}\PY{l+m+mi}{1}\PY{p}{]}\PY{o}{.}\PY{n}{set\PYZus{}title}\PY{p}{(}\PY{l+s+s1}{\PYZsq{}}\PY{l+s+s1}{Histogram of GDP growth rates}\PY{l+s+s1}{\PYZsq{}}\PY{p}{)}
         \PY{n}{ax}\PY{p}{[}\PY{l+m+mi}{1}\PY{p}{]}\PY{o}{.}\PY{n}{spines}\PY{p}{[}\PY{l+s+s1}{\PYZsq{}}\PY{l+s+s1}{right}\PY{l+s+s1}{\PYZsq{}}\PY{p}{]}\PY{o}{.}\PY{n}{set\PYZus{}visible}\PY{p}{(}\PY{k+kc}{False}\PY{p}{)} \PY{c+c1}{\PYZsh{} get ride of the line on the right}
         \PY{n}{ax}\PY{p}{[}\PY{l+m+mi}{1}\PY{p}{]}\PY{o}{.}\PY{n}{spines}\PY{p}{[}\PY{l+s+s1}{\PYZsq{}}\PY{l+s+s1}{top}\PY{l+s+s1}{\PYZsq{}}\PY{p}{]}\PY{o}{.}\PY{n}{set\PYZus{}visible}\PY{p}{(}\PY{k+kc}{False}\PY{p}{)}   \PY{c+c1}{\PYZsh{} get rid of the line on top}
         
         \PY{n}{plt}\PY{o}{.}\PY{n}{savefig}\PY{p}{(}\PY{l+s+s1}{\PYZsq{}}\PY{l+s+s1}{double.pdf}\PY{l+s+s1}{\PYZsq{}}\PY{p}{)}
         \PY{n}{plt}\PY{o}{.}\PY{n}{show}\PY{p}{(}\PY{p}{)}
\end{Verbatim}


    \begin{center}
    \adjustimage{max size={0.9\linewidth}{0.9\paperheight}}{output_39_0.png}
    \end{center}
    { \hspace*{\fill} \\}
    
    You can imagine how useful this can be. We can loop over sets of axes
and automate making plots if we have several variables.

I changed a couple other things here, too. 1. I used the
\texttt{figsize} parameter to subplot. This is a tuple of figure width
and height in inches. (Inches! Take that rest of the world!) The height
and width are of the \textbf{printed} figure. You will notice that
jupyter notebook scaled it down for display. This is useful when you are
preparing graphics for a publication and you need to meet an exact
figure size. 2. I made the line color a variable, so it is easy to
change all the line colors at one. For example, I like red figures when
I am giving presentations, but black figures when I am creating pdfs
that will be printed out on a black and white printer.


    % Add a bibliography block to the postdoc
    
    
    
    \end{document}
